\chapter{Arduino-Yún}
\label{chap:anexo3}

\section{Visión Global}

El Arduino Yún es una placa  electrónica que fusiona dos entornos para
ofrecer  lo mejor  de  ambos. Por  una parte  tenemos  un entorno  con
distribución  Linux  basado  en  OpenWrt  denominado  OpenWrt-Yún  que
incorpora  un  procesador  Atheros   AR9331,  que  permite  todas  las
funcionalidades  básicas  de un  sistema  operativo.   Por otra  parte
tenemos  el microcontrolador  ATmega32u4 que,  al igual  que en  otros
modelos de  Arduino, se  encarga de  la ejecución  de subrutinas  y el
control de los canales  de entrada/salida, solo que  en este caso la
placa cuenta con  un bridge de comunicaciones que  confiere al entorno
Arduino  la capacidad  ejecutar  scripts shells,  comunicarse con  las
interfaces de red y recibir información de su entorno vecino.

\section{Especificaciones Técnicas}

\begin{table}[h!]
  \centering
  \caption{Características Arduino-Yún. \cite{Arduino-Yun} }
  \label{tab:arduino-yun}
  \begin{tabular}{{p{0.325\linewidth}p{0.325\linewidth}p{0.325\linewidth}}}
    \hline
    \multicolumn{1}{l}{} & \textbf{Característica} & \textbf{Valor}\\
    \hline
      \multirow{6}{*}{Entorno Linux}  & \cellcolor[HTML]{E6E6E6} Procesador & \cellcolor[HTML]{E6E6E6} Atheros AR9331\\
                                      & Arquitectura & MIPS @400 MHz \\
                                      & \cellcolor[HTML]{E6E6E6} RAM & \cellcolor[HTML]{E6E6E6} 64MB DDR2 \\
                                      & Memoria FLASH & 16 MB \\
                                      & \cellcolor[HTML]{E6E6E6} Ethernet & \cellcolor[HTML]{E6E6E6} IEEE 802.3 10/100Mbit/s \\
                                      & WiFi & IEEE 802.11b/g/n \\
                                      \hline
    \multirow{6}{*}{Entorno Arduino}  & \cellcolor[HTML]{E6E6E6} Microcontrolador & \cellcolor[HTML]{E6E6E6} ATmega32u4\\
                                      &  Frecuencia Reloj & 16 MHZ \\
                                      & \cellcolor[HTML]{E6E6E6} SRAM & \cellcolor[HTML]{E6E6E6} 2.5 KB \\
                                      &  Memoria FLASH & 32 KB \\
                                      &  \cellcolor[HTML]{E6E6E6} EEPROM & \cellcolor[HTML]{E6E6E6} 1 KB \\
                                      & Entradas Analógicas & 20 \\
  \end{tabular}
\end{table}