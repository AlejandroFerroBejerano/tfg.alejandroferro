\chapter{Resumen}


Aunque en los tiempos actuales las nuevas tecnologías son absorbidas y aplicadas en campos que hasta el momento no eran habituales, ofreciendo nuevos enfoques y posibilidades, la aplicación de estas a los sistemas de seguridad sigue siendo muy lenta comparada con sectores como la telefonía móvil, el desarrollo de sistemas operativos, aplicaciones, etc.

Además, hay que tener en cuenta que a día de hoy podemos encontrar toda clase de software y hardware libre, que abarcan desde sistemas operativos a impresoras tridimensionales de bajo coste y sin embargo, no encontramos ningún sistema de seguridad física en el mercado que cumpla con estas características. Por el contrario, podemos encontrar decenas de soluciones comerciales que se presentan como cajas negras, donde no se divulgan ni siquiera las tecnologías que se utilizan y donde muchas veces la fase de \textit{testing} no es todo lo estricta que debiera ser y se concluye sobre los sistemas en producción del propio cliente.

Por estos motivos,  el presente Trabajo Fin de Grado se centrará en evaluar y ofrecer una solución que dé respuesta a estas carencias, a modo de prototipo funcional, permitiendo supervisar dispositivos de intrusión de \acf{GII} o \acf{GIII} de una forma cómoda y sencilla para los usuarios que lo operen. Además, la problemática abordada y las soluciones que este trabajo propone, se han expuesto a la comunidad universitaria y del software libre. Para ello se  ha presentado este proyecto al \textbf{XI Concurso Universitario de Software Libre}, organizado en la universidad de Sevilla, consiguiendo no solo el objetivo descrito anteriormente, sino también el reconocimiento al trabajo realizado a través del premio: \textbf{Mejor Proyecto de Seguridad} del certamen.
 



\chapter{Abstract}

Although new technologies are being absorbed and applied to fields that have hitherto been unusual, providing new approaches and possibilities, the penetration of these \mbox{technologies} in security systems is still very slow compared to sectors such as mobile telephony, \mbox{development} of operating systems, applications, etc.

Nowadays, we can find a wide variety of free software and hardware resources, \mbox{ranging} from operating systems to low-cost \mbox{three-dimensional} printers, and yet we do not find any physicall security system in the market that meets these characteristics. However, there is not any physical security system that, adopting the open source philosophy, can be found in the market. On the contrary, we can find dozens of \mbox{commercial} solutions that are \mbox{presented} as black boxes, which even hide the underlying technologies and whose \textit{testing} phase is not as thorough as it should be and is concluded over the systems of the own client in \mbox{production}. 

For this reason the present Bachelor Thesis will focus this work on evaluating and \mbox{providing} a solution that solves these shortcomings, as a functional prototype, allowing the \mbox{monitoring} of intrusion devices of types \acs{GII} or \acs{GIII} in a convenient and simple way for its users. \mbox{Moreover},  we have tried to expose this reality and the solutions described in this report to the university \mbox{community} and free software communities. For this purpose, the project was submitted to the \textbf{XI Free-Software \mbox{University} Contest}, organized at the University of \mbox{Seville}, \mbox{achieving} not only the objective described above, but also the recognition of the work through the prize: \textbf{Best Security Project} of the contest.
