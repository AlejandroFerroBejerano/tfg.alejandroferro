\chapter{Agradecimientos}

Después de cinco años de carrera y una pausa académica de año y medio por motivos profesionales, hoy es el día en el que cruzo la meta (ya iba siendo hora) y me embarco en otra maravillosa travesía como profesional y persona.

Aunque es difícil recordar a todos los que me han ayudado voy a a hacer un último esfuerzo (literalmente hablando, el último) para hacer justicia a todos los que han aportado su granito o montón de harina a este costal.

A gradecer a mis dos directores de proyecto Maria José Santofimia y David Villa por su infinita paciencia, dedicación, guía y aguante hasta el último momento. Mariajo muchas gracias por hacerme un hueco en tu agenda a pesar de fijar las tutorías en días festivos, haber sido recientemente mamá o tener mil historias que hacer en el trabajo. David muchas gracias por animarme a elegir este tema de trabajo y por ser una gran fuente de soporte y conocimientos técnicos aportados al proyecto.

Como no podía ser de otra forma, agradecer enormemente a la banca; mi madre y Juanjo. Sin vosotros esta aventura que ha sido estudiar una carrera universitaria no se hubiese podido completar satisfactoriamente, no solo por la financiación, también por vuestra confianza y apoyo. 

Gracias a mi padre y hermanos por creer siempre en mí y por los ánimos que me han dado durante todo este tiempo. Viajar 7.441,68 km para estar el día de la defensa de mi \acs{TFG} os avalan.

Agradecer a Desidero, Alicia, Celia, la abuela Amalia y Paco (los patrocinadores) por acogerme en su familia y por apoyarme incondicionalmente. Sin vosotros hubiese sido todo mucho más difícil.

Agradecer a mis dos soles Chicho y Mamita por estar siempre ahí, con disponibilidad completa para cuando os he necesitado (literalmente). Una mirada vuestra ha sido suficiente para infundirme ánimos cuando me han flaqueado las fuerzas.

Agradecer a mi compañera de viaje y futura esposa Diana, que ha puesto su corazón, sudor y esfuerzo para que este día llegase y que siempre tuvo la palabra correcta para infundir en mí lo que la situación requería.

Gracias a mi amiga Arianna Pérez por esas terapias tan buenas y positivas que me dió, sin esos ánimos quizás no me hubiese decidido a venir a Ciudad Real y retomar la vida de estudiante.

A mis amigos de la carrera: Mig, Javi, Juanma, Marcos, Kike, Ricky, Claudio, Mariajo  y todos los que no alcanzaría a nombrar porque seriáis muchos. Gracias por ser el complemento perfecto al trabajo y por haber dado color a la aventura. 

Agradecer a mi amigo y mentor Alejandro Moreno por todos los conocimientos que me ha ido aportando a través de los años en el sector de la seguridad y por recibirme en su casa como a uno más.

Gracias a mis compañeros y amigos del grupo de investigación Arco: Manu, Tato y Dani por ayudarme siempre en lo que os pedí y por haberme mostrado cada uno una visión diferente de lo que significa ser ingeniero informático.

Un agradecimiento especial a todo el equipo docente que estuvo batallando conmigo a lo largo de estos cinco años. Gracias por todas las enseñanzas y experiencias transmitidas.

Y finalmente, aunque no por ello menos importante y porque en todos los caminos existen piedras con las que tropezamos y nos levantamos. Agradecer a todas aquellas personas que siempre dudaron de mi y pensaron que jamás lo lograría. Nada ha hecho tan fuerte al necio que llevo dentro y me ha dado tanta tenacidad para conseguir este objetivo como esos pensamientos.

Gracias a todos.

\quoteauthor{Alejandro Julián Ferro Bejerano}
