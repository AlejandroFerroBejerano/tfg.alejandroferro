\chapter{Arduino LyliPad}
\label{chap:anexo2}

\section{Visión Global}
El valor añadido de Arduino Lilypad, gracias al potencial del hardware Arduino, es la capacidad que da a los tejidos de detectar información sobre el entorno, mediante el uso de sensores de luz, movimiento o temperatura. Eso permite al tejido ofrecer respuestas ante los cambios ambientales pudiendo encenderse como un árbol de navidad gracias a las luces LED, funcionar ante las vibraciones o generar sonido ante un estímulo. Su procesador se basa en el chip ATmega168V (la versión de bajo consumo) o el chip ATmega328V . 

\section{Especificaciones Técnicas}

 \begin{table}[h!]
  \centering
  \caption{Características Arduino LilyPad. \cite{ArduinoLilyPad}}
  \label{tab:LilyPad}
  \zebrarows{1}
  \begin{tabular}{p{0.48\textwidth}p{0.48\textwidth}}
    \hline
    \textbf{Característica} & \textbf{Valor} \\
    \hline
    Microcontrolador & ATmega168 o ATmega328V \\
    Tensión de funcionamiento & 2.7 a 5.5 V \\
    Voltaje de entrada (recomendado) & 2.7 a 5.5 V\\
    Entradas/Salidas digitales & 14 \\
    Canales PWM & 6 \\
    Pines de entrada analógica & 6 \\
    Corriente continua para Pin I/O & 40mA \\
    Memoria flash & 16 KB, 2 KB utilizado por el gestor de arranque \\
    SRAM & 1 KB \\
    EEPROM & 512 bytes \\
    Frecuencia de reloj & 8 MHz \\
    \hline
  \end{tabular}
\end{table}